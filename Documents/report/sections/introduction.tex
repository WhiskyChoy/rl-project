\section{Introduction}
In the past 50 years, we've seen many severe aviation accidents, while among which the collision of two aircraft in the air is considered the most appalling ones, as they caused the worst of the losses for both sides of the collision. According to European Union Aviation Safety Agency (EASA), such accidents can be defined as mid-air collision \citep{de2011use}. To avoid mid-air collision, the concepts of air traffic confict and safety zone are set by specialized agencies. In China, two aircraft are considered in-conflict if they had a distance less than 5 nautical miles horizontally and 300 meters vertically, and such space that keep safety separation is defined as the safety zone. With these concepts introduced, there are actually more cases of air traffic conflict than mid-air collision although the planes did not collide. To avoid mid-air collision, currently people rely on air traffic control (ATC) to reduce air traffic confict. With the fast growth of global air traffic density and complexity, it becomes a big challenge to ensure effective ATC. However, nowadays tatical decisions are still being made by human air traffic controllers, with the mechanism of which remains largely the same as that of 50 years ago \citep{national2014autonomy}.

The density could increase a lot when it comes to low-altitude airspace traffic, while the behaviors of aircraft become more complex \citep{kopardekar2015safely}. What's more, the vertical separation distance is harder to maintain and one airway can no longer contain as many flight levels as that in high-altitude airspace due to the limited ground clearance. One solution is to strictly divide aircraft activities into disjoint space, but this could cause low throughput which really matters at high traffic density. Therefore, some strategy like time-division multiplexing, which allows the aircraft to pass the same space at different time is inevitably adopted \citep{mueller2017enabling}. In such cases, we need real-time autonomous ATC systems to decide the aircraft actions, most possibly to change the velocity, to avoid potential mid-air collision in some specific sector.

For real-time autonomous ATC within high-density airspace, traditional optimizing methods might not be ideal solutions. To get optimal solution under such conditions, traditional approaches require global information of all aircraft in the sector and made centralized decisions\citep{farley2007fast}. Note that such process is rather time-consuming, and it also takes time for aircraft to make reactions to the orders given by a central controller like the airport traffic control tower. Considering that the process is latency-sensitive, the approach of reinforcement learning could be a better choice. Although it's slow to train a reinforcement learning model, it's very quick for the model to make interpretation \citep{brittain2019autonomous}. Therefore, it's applicable for each aircraft to use reinforcement learning methods to make instant decision. What's more, as long as the model is pretrained before loaded into each aircraft, the procedure of making en-route advisories will not cause unacceptable computational pressure.

The structure of the paper is organized as follows. Section 2 reviews the background and  related work. In Section 3, our detailed models, proposed algorithm and environment settings will be introduced. Section 4 presents our experiment results via our designed framework, followed by conclusions and future works in Section 5. Lastly, in the appendix, we briefly introduce the schedule and contribution, and the repository of our project on \emph{Github}.